\documentclass{article}
\usepackage{graphicx}
\usepackage{amsmath}
\usepackage{array}
\usepackage{fancyhdr}
\usepackage{amssymb}
\usepackage[shortlabels]{enumitem}

\DeclareMathOperator{\R}{\mathbb R}

\pagestyle{fancy}
\fancyhead[L]{Banghao Chi}
\fancyhead[C]{Homework 3}
\fancyhead[R]{11th Feb}

\fancyfoot[C]{\thepage}

\renewcommand{\headrulewidth}{0.5pt}
\renewcommand{\footrulewidth}{0.5pt}

\begin{document}

\section*{Exercise 1}
(10 points) Consider the dot product $\cdot : \mathbb{R}^n \times \mathbb{R}^n \to \mathbb{R}$, and its associated (Euclidean) norm $\|\cdot\|$. Let $B$ be the closed ball of radius 1 centered at the origin; that is,
\[
    B = \{\mathbf{x} \in \mathbb{R}^n : \|\mathbf{x}\| \leq 1\}.
\]
Prove the following statement, which shows that $B$ is an intersection of affine half-spaces:
\[
    B = \bigcap_{\substack{\mathbf{a}\in\mathbb{R}^n \\ \|\mathbf{a}\|=1}} \{\mathbf{x} \in \mathbb{R}^n : \mathbf{a}^T\mathbf{x} \leq 1\}.
\]

\textbf{Solution:}

\newpage

\section*{Exercise 2}
(20 points) Use Jensen's inequality to find a solution for each of the following problems. (In each problem your answer should be a pair $(x,y)$ or a triple $(x,y,z)$.)
\begin{itemize}
\item Minimize $e^x + e^y + 2e^z$ subject to $\frac{1}{4}x + \frac{1}{4}y + \frac{1}{2}z = 2$.
\item Minimize $7 + \sqrt[3]{e^x + e^{2y}}$ subject to $2x + 4y = 9$.
\item Maximize $xyz$ subject to $x + y + z = 4$ and $x,y,z > 0$.
\item Minimize $16^x + 2^{y+2}$ subject to $x + y = 7$.
\end{itemize}

\textbf{Solution:}

\newpage

\section*{Exercise 3}
(10 points) Let $X$ be a random variable that takes values in a finite set $S$, where $S$ is a subset of the interval $[\frac{\pi}{4}, \frac{2\pi}{5}]$. Recall that the expected value of $X$ is defined as
\[
    E[X] = \sum_{s\in S} s\Pr(X = s).
\]
Suppose that $E[X] = \frac{\pi}{3}$. Prove that the expected value of the cosine of $X$ is at most $\frac{1}{2}$.

Hint: A function $f$ is concave if $-f$ is convex. One can show that a concave function satisfies Jensen's inequality, but with the inequality reversed.

\textbf{Solution:}

\newpage

\end{document}