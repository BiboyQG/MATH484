\documentclass{article}
\usepackage{graphicx}
\usepackage{amsmath}
\usepackage{array}
\usepackage{fancyhdr}
\usepackage{amssymb}
\usepackage[shortlabels]{enumitem}

\DeclareMathOperator{\R}{\mathbb R}

\pagestyle{fancy}
\fancyhead[L]{Banghao Chi}
\fancyhead[C]{Homework 3}
\fancyhead[R]{11th Feb}

\fancyfoot[C]{\thepage}

\renewcommand{\headrulewidth}{0.5pt}
\renewcommand{\footrulewidth}{0.5pt}

\begin{document}

\section*{Exercise 1}
(10 points) Consider the dot product $\cdot : \mathbb{R}^n \times \mathbb{R}^n \to \mathbb{R}$, and its associated (Euclidean) norm $\|\cdot\|$. Let $B$ be the closed ball of radius 1 centered at the origin; that is,
\[
    B = \{\mathbf{x} \in \mathbb{R}^n : \|\mathbf{x}\| \leq 1\}.
\]
Prove the following statement, which shows that $B$ is an intersection of affine half-spaces:
\[
    B = \bigcap_{\substack{\mathbf{a}\in\mathbb{R}^n \\ \|\mathbf{a}\|=1}} \{\mathbf{x} \in \mathbb{R}^n : \mathbf{a}^T\mathbf{x} \leq 1\}.
\]

\textbf{Solution:} \\

Since this is an equality, we need to show both inclusion. \\

$\Rightarrow$: \\

Let $\mathbf{x} \in B$. Then $\|\mathbf{x}\| \leq 1$.
For any unit vector $\mathbf{a}$ (i.e., $\|\mathbf{a}\| = 1$), by the Cauchy-Schwarz inequality:
\[
    \mathbf{a}^T\mathbf{x} \leq |\mathbf{a}^T\mathbf{x}| \leq \|\mathbf{a}\|\|\mathbf{x}\| \leq 1
\]

Therefore, $B \subseteq \bigcap_{\substack{\mathbf{a}\in\mathbb{R}^n \\ \|\mathbf{a}\|=1}} \{\mathbf{x} \in \mathbb{R}^n : \mathbf{a}^T\mathbf{x} \leq 1\}$. \\

$\Leftarrow$: \\

Let $\mathbf{x}$ be in the intersection. We need to show $\|\mathbf{x}\| \leq 1$.

\begin{itemize}
    \item If $\mathbf{x} = \mathbf{0}$, then clearly $\|\mathbf{x}\| = 0 \leq 1$.
    \item If $\mathbf{x} \neq \mathbf{0}$, consider $\mathbf{a} = \frac{\mathbf{x}}{\|\mathbf{x}\|}$, $\|\mathbf{a}\| = 1$.
\end{itemize}

Since $\mathbf{x}$ is in the intersection:
\[
    \mathbf{a}^T\mathbf{x} = \frac{\mathbf{x}^T\mathbf{x}}{\|\mathbf{x}\|} = \|\mathbf{x}\| \leq 1
\]

Therefore, $\bigcap_{\substack{\mathbf{a}\in\mathbb{R}^n \\ \|\mathbf{a}\|=1}} \{\mathbf{x} \in \mathbb{R}^n : \mathbf{a}^T\mathbf{x} \leq 1\} \subseteq B$. \\

Therefore,
\[
    B = \bigcap_{\substack{\mathbf{a}\in\mathbb{R}^n \\ \|\mathbf{a}\|=1}} \{\mathbf{x} \in \mathbb{R}^n : \mathbf{a}^T\mathbf{x} \leq 1\}
\]

\newpage

\section*{Exercise 2}
(20 points) Use Jensen's inequality to find a solution for each of the following problems. (In each problem your answer should be a pair $(x,y)$ or a triple $(x,y,z)$.)
\begin{itemize}
\item Minimize $e^x + e^y + 2e^z$ subject to $\frac{1}{4}x + \frac{1}{4}y + \frac{1}{2}z = 2$.
\item Minimize $7 + \sqrt[3]{e^x + e^{2y}}$ subject to $2x + 4y = 9$.
\item Maximize $xyz$ subject to $x + y + z = 4$ and $x,y,z > 0$.
\item Minimize $16^x + 2^{y+2}$ subject to $x + y = 7$.
\end{itemize}

\textbf{Solution:} \\

(1) Minimize $e^x + e^y + 2e^z$ subject to $\frac{1}{4}x + \frac{1}{4}y + \frac{1}{2}z = 2$ \\

By Jensen's inequality, since $e^x$ is convex:
\[
    \frac{1}{4}e^x + \frac{1}{4}e^y + \frac{2}{4}e^z \geq e^{\frac{1}{4}x + \frac{1}{4}y + \frac{2}{4}z} = e^2
\]

Therefore:
\[
    e^x + e^y + 2e^z \geq 4e^2
\]

Equality holds when $x = y = z$, using the constraint:
\[
    \frac{1}{4}x + \frac{1}{4}y + \frac{1}{2}z = 2 \implies \frac{1}{4}x + \frac{1}{4}x + \frac{1}{2}x = 2 \implies x = 2
\]

Therefore, the optimal solution is $(2,2,2)$ \\

(2) Minimize $7 + \sqrt[3]{e^x + e^{2y}}$ subject to $2x + 4y = 9$ \\

By Jensen's inequality, since $e^x$ is convex:
\[
    7 + \sqrt[3]{e^x + e^{2y}} = 7 + \sqrt[3]{2(\frac{1}{2}e^x + \frac{1}{2}e^{2y})} \geq 7 + \sqrt[3]{2e^{\frac{1}{2}x + y}} = 7 + \sqrt[3]{2e^{\frac{9}{4}}}
\]

Equality holds when $x = 2y$, using the constraint:
\[
    2x + 4y = 9 \implies 2(2y) + 4y = 9 \implies 4y + 4y = 9 \implies y = \frac{9}{8}
\]

Therefore, the optimal solution is $(\frac{9}{4}, \frac{9}{8})$ \\

(3) Maximize $xyz$ subject to $x + y + z = 4$ and $x,y,z > 0$ \\

The problem is equivalent to minimizing $-log(xyz) = -log(x) - log(y) - log(z)$ subject to $x + y + z = 4$ and $x,y,z > 0$, since $-log(x)$ is strictly decreasing. \\

By Jensen's inequality, since $-log(x)$ is convex:
\[
    -(\frac{1}{3}log(x) + \frac{1}{3}log(y) + \frac{1}{3}log(z)) \geq -log(\frac{x + y + z}{3}) = -log(\frac{4}{3})
\]

Therefore:
\[
    -log(xyz) \geq 3(-log(\frac{4}{3})) = -3log(\frac{4}{3})
\]

Equality holds when $x = y = z$, using the constraint:
\begin{equation*}
    x + y + z = 4 \implies 3x = 4 \implies x = \frac{4}{3}
\end{equation*}

Therefore, the optimal solution is $(\frac{4}{3}, \frac{4}{3}, \frac{4}{3})$ \\

(4) Minimize $16^x + 2^{y+2}$ subject to $x + y = 7$ \\

Let $u = 16^x = (2^4)^x = 2^{4x}$ and $v = 2^{y+2} = 2^{y+2}$ \\

Since $2^t$ is convex and $x + y = 7$:
\[
    2^{4x} + 2^{y+2} = 5(\frac{1}{5}2^{4x} + \frac{4}{5}2^{y}) \geq 5(2^{\frac{4}{5}x + \frac{4}{5}y}) = 5(2^{\frac{4}{5}(x + y)}) = 5(2^{\frac{4}{5}(7)}) = 5(2^{\frac{28}{5}})
\]

Equality holds when $4x = y$, using the constraint:
\[
    x + y = 7 \implies x + 4x = 7 \implies 5x = 7 \implies x = \frac{7}{5}
\]

Therefore, the optimal solution is $(\frac{7}{5}, \frac{28}{5})$

\newpage

\section*{Exercise 3}
(10 points) Let $X$ be a random variable that takes values in a finite set $S$, where $S$ is a subset of the interval $[\frac{\pi}{4}, \frac{2\pi}{5}]$. Recall that the expected value of $X$ is defined as
\[
    E[X] = \sum_{s\in S} s\Pr(X = s).
\]
Suppose that $E[X] = \frac{\pi}{3}$. Prove that the expected value of the cosine of $X$ is at most $\frac{1}{2}$.

Hint: A function $f$ is concave if $-f$ is convex. One can show that a concave function satisfies Jensen's inequality, but with the inequality reversed. \\

\textbf{Solution:} \\

cosine is a concave function on the interval $[\frac{\pi}{4}, \frac{2\pi}{5}]$, which can be proved by showing that $-cos(x)$ is convex:
$$(-cos(x))'' = cos(x) \geq 0 \text{ on } [\frac{\pi}{4}, \frac{2\pi}{5}]$$

For a concave function $g$ and a random variable $Y$, Jensen's inequality also holds in the following form:
   \[
   E[g(Y)] \leq g(E[Y])
   \]

Applying this to our problem where $g(x) = \cos(x)$ and $Y = X$:
\[
    E[\cos(X)] \leq \cos(E[X]) = \cos(\frac{\pi}{3}) = \frac{1}{2}
\]

Therefore:
\[
    E[\cos(X)] \leq \frac{1}{2}
\]

\newpage

\end{document}