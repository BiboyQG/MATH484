\documentclass{article}
\usepackage{graphicx}
\usepackage{amsmath}
\usepackage{array}
\usepackage{fancyhdr}
\usepackage{amssymb}
\usepackage[shortlabels]{enumitem}

\DeclareMathOperator{\R}{\mathbb R}

\pagestyle{fancy}
\fancyhead[L]{Banghao Chi}
\fancyhead[C]{Homework 7}
\fancyhead[R]{11th Apr}

\fancyfoot[C]{\thepage}

\renewcommand{\headrulewidth}{0.5pt}
\renewcommand{\footrulewidth}{0.5pt}

\begin{document}

\section*{Exercise 1}
(5 points) Let $f$ be a function of the form $f(x) = ax + bx^3$, where $a, b \in \mathbb{R}$. Suppose that $a$ and $b$ are chosen so that
\begin{align*}
\text{error} = 3(f(-1) - 1)^2 + (f(0) - 0)^2 + (f(1) - 1)^2 + (f(2) - 3)^2
\end{align*}
is minimized. (In other words, $f$ approximates the data points $(-1,1)$, $(0,0)$, $(1,1)$, $(2,3)$, so that the error squared at $x = -1$ is given three times as much weight as other errors squared.) Write a matrix expression for the pair $(a, b)$ that minimizes the \text{error} expression. \\

\textbf{Solution:}
\begin{align*}
f(-1) &= -a - b \\
f(0) &= 0 \\
f(1) &= a + b \\
f(2) &= 2a + 8b
\end{align*}

Therefore, we have matrix $A$:
\begin{align*}
A = \begin{pmatrix} 
-1 & -1 \\ 
0 & 0 \\ 
1 & 1 \\ 
2 & 8
\end{pmatrix}
\end{align*}

The target values form vector $\mathbf{b}$:
\begin{align*}
\mathbf{b} = \begin{pmatrix} 
1 \\ 
0 \\ 
1 \\ 
3
\end{pmatrix}
\end{align*}

Since the error at $x = -1$ is given three times as much weight, the weight matrix $H$ is:
\begin{align*}
H = \begin{pmatrix} 
3 & 0 & 0 & 0 \\ 
0 & 1 & 0 & 0 \\ 
0 & 0 & 1 & 0 \\ 
0 & 0 & 0 & 1
\end{pmatrix}
\end{align*}

According to the theorem, the solution is:
\begin{align*}
\mathbf{x}^* = (A^T H A)^{-1} A^T H \mathbf{b}
\end{align*}

\newpage

\section*{Exercise 2}
(10 points) Consider the convex program
\begin{align*}
(P_c) \begin{cases}
\underset{x,y\in\mathbb{R}}{\text{minimize}} & x + y \\
\text{subject to} & y^2 - x \leq 0, \\
& x - y - c \leq 0.
\end{cases}
\end{align*}
For which values of $c$ is $P_c$ superconsistent? \\

\textbf{Solution:}
\begin{align*}
y^2 - x &< 0\\
x - y - c &< 0
\end{align*}

Rearranging these constraints:
\begin{align*}
x &> y^2\\
x &< y + c\\
\Rightarrow y^2 - y - c &< 0
\end{align*}

Therefore:
\begin{align*}
\left(y - \frac{1}{2}\right)^2 &< \frac{1}{4} + c
\end{align*}

This inequality has solutions if and only if $\frac{1}{4} + c > 0$, or equivalently, $c > -\frac{1}{4}$. \\

Therefore, $P_c$ is superconsistent if and only if $c > -\frac{1}{4}$.

\newpage

\section*{Exercise 3}
(10 points) For the convex program
\begin{align*}
(P) \begin{cases}
\underset{x\in\mathbb{R}}{\text{minimize}} & x^2 \\
\text{subject to} & x^2 \leq 0, \\
& -x \leq 0
\end{cases}
\end{align*}
find the domain of the function $MP(z_1, z_2)$. Then, express the function $MP(z_1, z_2)$ explicitly in terms of $z_1$ and $z_2$. \\

\textbf{Solution:}

\begin{itemize}
    \item From the first constraint $x^2 \leq z_1$:
    \begin{itemize}
        \item Since $x^2 \geq 0$ for all $x \in \mathbb{R}$, $z_1 \geq 0$, and the constraint becomes $-\sqrt{z_1} \leq x \leq \sqrt{z_1}$.
    \end{itemize}
    \item From the second constraint $-x \leq z_2$:
    \begin{itemize}
        \item This is equivalent to $x \geq -z_2$.
    \end{itemize}
\end{itemize}

\text{For the problem to be feasible, the intervals $[-\sqrt{z_1}, \sqrt{z_1}]$ and $[-z_2, \infty)$ must intersect.} \\

\text{So we have $z_2 \geq -\sqrt{z_1}$.} \text{Therefore, the domain of $MP(z_1, z_2)$ is:} $$\{(z_1, z_2) : z_1 \geq 0, z_2 \geq -\sqrt{z_1}\}$$

\text{For $(z_1, z_2)$ in the domain, since we're minimizing $x^2$, we want $x$ to be as close to 0 as possible.}

\begin{itemize}
    \item Case 1: If $z_2 \geq 0$, then $-z_2 \leq 0$.
    \begin{itemize}
        \item Since 0 is in the feasible region, the minimum value is $x^2 = 0$ at $x = 0$.
    \end{itemize}
    \item Case 2: If $z_2 < 0$, then $-z_2 > 0$.
    \begin{itemize}
        \item The minimum value of $x^2$ would occur at $x = -z_2$.
    \end{itemize}
\end{itemize}

\text{Therefore:}
\begin{align*}
MP(z_1, z_2) = 
\begin{cases}
0, & \text{if } z_2 \geq 0 \\
z_2^2, & \text{if } z_2 < 0
\end{cases}
\end{align*}

\newpage

\section*{Exercise 4}
(15 points) Let $P$ be a superconsistent convex program in standard form.
\begin{itemize}
\item Explain why if $MP(\mathbf{z})$ is differentiable at $\mathbf{z} = \mathbf{0}$, then $\boldsymbol{\lambda}^* = -\nabla MP(\mathbf{0})$ is a sensitivity vector of $P$. (In fact, $\boldsymbol{\lambda}^* = -\nabla MP(\mathbf{0})$ is the only sensitivity vector.)
\item Find a sensitivity vector $\boldsymbol{\lambda}^*$ for the program
\begin{align*}
(P) \begin{cases}
\underset{x\in\mathbb{R}}{\text{minimize}} & (x - 1)^2 + (y - 3)^2 \\
\text{subject to} & x \leq 0, \\
& y \leq 0
\end{cases}
\end{align*}
\end{itemize}

\textbf{Solution:} \\

Given that $MP(\mathbf{z})$ is differentiable at $\mathbf{z} = \mathbf{0}$, we can consider small perturbations $\Delta \mathbf{z}$. By the definition of differentiability:
\begin{align*}
MP(\Delta\mathbf{z}) = MP(\mathbf{0} + \Delta\mathbf{z}) \approx MP(\mathbf{0}) + \nabla MP(\mathbf{0})^{\top} \Delta\mathbf{z}
\end{align*}

This gives us:
\begin{align*}
MP(\mathbf{0} + \Delta\mathbf{z}) - MP(\mathbf{0}) \approx \nabla MP(\mathbf{0})^{\top} \Delta\mathbf{z}
\end{align*}

If we define $\boldsymbol{\lambda}^* := -\nabla MP(\mathbf{0})$, then:
\begin{align*}
MP(\mathbf{0} + \Delta\mathbf{z}) - MP(\mathbf{0}) \approx -\boldsymbol{\lambda}^{*\top} \Delta\mathbf{z}
\end{align*}

By definition, this satisfies the definition of a sensitivity vector. Therefore, $\boldsymbol{\lambda}^* = -\nabla MP(\mathbf{0})$ is one possible sensitivity vector for the problem P. \\

For the program:
\begin{align*}
(P) \begin{cases}
\underset{x,y\in\mathbb{R}}{\text{minimize}} & (x - 1)^2 + (y - 3)^2 \\
\text{subject to} & x \leq 0 \\
& y \leq 0
\end{cases}
\end{align*}

$P(z_1, z_2)$ is:
\begin{align*}
\text{minimize} \quad & (x - 1)^2 + (y - 3)^2 \\
\text{subject to} \quad & x \leq z_1 \\
& y \leq z_2
\end{align*}

For $\mathbf{z} = (z_1, z_2)$ near the origin (for $z_1 < 1$ and $z_2 < 3$), the optimal solution is $(z_1, z_2)$, and:
\begin{align*}
MP(\mathbf{z}) = (z_1 - 1)^2 + (z_2 - 3)^2
\end{align*}
\begin{align*}
\frac{\partial MP}{\partial z_1}(0) &= 2(0 - 1) = -2 \\
\frac{\partial MP}{\partial z_2}(0) &= 2(0 - 3) = -6
\end{align*}

So $\nabla MP(\mathbf{0}) = (-2, -6)$. \\

Therefore, the sensitivity vector is:
\begin{align*}
\boldsymbol{\lambda}^* = -\nabla MP(\mathbf{0}) = (2, 6)
\end{align*}

\end{document}